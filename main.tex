\title{Rhizosphere metagenomics of three biofuel crops}
\author{}

\documentclass[12pt]{article}
\usepackage[a4paper, margin=1in]{geometry}
\usepackage[parfill]{parskip}

% set double space
\usepackage{setspace}
\doublespacing

% set times fond
\usepackage{times}

% line number index
\usepackage{lineno}
\linenumbers

% Compile only with pdfLaTeX
\usepackage[pdftex]{graphicx}

% nice table format
\usepackage{booktabs}

% caption adjustment
\usepackage{caption}
\captionsetup[table]{singlelinecheck=off}

\begin{document}
\maketitle
\section{Introduction}

- Bioenergy and sustainability

Bioenergy is green energy, but growing bioenergy crop is not fully environmental friendly. With appropriate management, it can however be managed to have positive environmental effects. One area of management where sustainable development could help increase benefits is the growth of bioenergy crops. Bioenergy crop growth requires energy inputs (fertilizer, pesticides for corn), emits greenhouse gases (CO2, N2O), and competes with food production for land 1. The rhizosphere, or the interface between plant roots and soils, is an especially important part of the soil where microbes and plants form beneficial associations. Rhizosphere microbes provide critical nutrients (N, P) to plants, produce hormones that promote plant growth, improve plant pathogen resistance, and improve the soil physical structure in addition to many yet unknown beneficial relationships 2–6. This beneficial supporting microbial community is important for sustainable plant growth and consequently identifying these beneficial microbes and their functions is important for low cost growth of bioenergy plants.

- Three crops, annual and perennial, agricultural management

Rhizosphere microbes are influenced by many factors, such as the particular vegetation, agricultural management inputs, soil type, moisture and other environmental factors. In several large-scale studies, microbial communities are usually clustered based on their location because soil and ecosystem type has long-term selection and thus stronger influence than other factors. Vegetation also has influence on microbial community structure through the root exudates and particulate residues, which is the primary annual carbon and energy source for soil microbes. However, their influence varies depending on the plant species and ecosystem type 7,9,10. Three biofuel crops (corn, switchgrass and miscanthus) are from the same family (true grasses), but they are very different in phenotype. Miscanthus and switchgrass are perennial grasses with fibrous roots, while corn is an annual with less fibrous root. Miscanthus is the largest in size and originated from tropical regions, while switchgrass is native in Midwest, US. Large scale planting of these biofuel crops may impact soil microbial community structure and function, including N cycle processes. Plants also take time to select their beneficial groups and stabilize them, so the time since the establishment is also an important factor. Since switchgrass and miscanthus can growth with no (or less) new N input 7,11, these two perennial grasses are thought to have better nitrogen sustainability. Moreover, miscanthus is big in size and thus thought to have a community more efficient at fixing and keeping nitrogen in the soil. My preliminary studies have shown that the corn selected community is more different from switchgrass and miscanthus, than the later are from each other. However, it is still not clear which microbial groups make the miscanthus rhizosphere microbial community more efficient with its N economy 11,12.

Other than vegetation, agricultural management of the vegetation such as tillage, liming, and fertilization, also influence the rhizosphere microbial community by changing soil physical and chemical characteristics 13,14. Application of nitrogen fertilizer should decrease N fixation and increase nitrification and denitrification because N fixing microbes are not selected in higher N environments while nitrifying and denitrifying bacteria should be favored since they have more substrate. Within the nitrifying community, AOB (ammonia-oxidizing bacteria) is more responsive to fertilization than AOA (ammonia- oxidizing archaea), though AOA is usually more abundant than AOB 8,15. In the management of these three biofuel crops, corn needs more nitrogen fertilizer while switchgrass and miscanthus needs very little. Nitrogen fixing microbes may play an important role providing nitrogen for growth of miscanthus. Previous studies have shown that sugarcane can have 40\% to 60\% of its nitrogen supply from bacterial nitrogen fixation 16,17. Thus miscanthus and corn, close relatives of sugarcane may also enrich nitrogen fixing bacteria. Another study based on amplicon sequencing and qPCR of a few N-cycle genes found the abundance of N-fixing genes (nifH) is significantly higher in miscanthus and switchgrass than corn 7. The microbial communities of miscanthus and switchgrass are adapted to a low N environment and more efficient at keeping nitrogen in the soil, so are predicted to have increased N fixation, and decreased nitrification and denitrification. In addition, species well adapted to low the N environment should increase.

In addition to fertilization, the difference on the ways these three crops re-grow may also make a difference in their microbial community. Corn re-grows from new seeds, while the two perennial grasses continue to grow from their existing crown/roots. I predict that corn will select the fast growers since they grow a new root system and select its rhizosphere community every year. The perennial should have a more stable rhizosphere community because the selected community from previous year is less disrupted.

- N cycle

Despite the limited understanding of rhizosphere microorganisms, the chemistry of the nutrient cycles carried out by those microorganisms is relatively clear, including nitrogen fixation, nitrification, and denitrification. Nitrogen fixing microbial community are commonly measured with nifH as marker and nitrifying community are commonly measured with AOA (archaeal amoA) and AOB (baterial amoA). Denitrification has multiple steps including nitrate reduction (Nar), nitrite reduction (Nir), nitric oxide reduction (Nor) and nitrous oxide reduction (Nos). Denitrifying community are commonly measured with Nir, Nor, and Nos. Nir has two types, Cu-Nir (copper based) and cd1-Nir (iron based), encoded by nirK and nirS respectively. Nor has cNor (with cytochrome c) and qNor (without cytochrome c). Nos also has two types, encoded by typical nosZ and atypical nosZ. Studies only survey one gene may underestimate microbes involved in denitrification. (ADD REF)

- Amplicon vs. shotgun metaG

Many previous studies have focused on identifying N-relevant genes in soils through PCR-amplified amplicons of targeted genes (SSU rRNA gene or N cycle genes) 7,12. However, primer bias is a well-known problem for amplicon-based analysis, especially for those N cycle genes that lack well curated reference databases 18–20. Here, I propose to use high throughput shotgun sequencing provided by Illumina Hiseq technology. All DNA in the community is randomly fragmented and sequenced to produce less biased metagenomes of microbial communities compared to gene-targeted amplicons. Further, rather than amplicons only targeting one gene or part of a gene, the shotgun method produces short fragments representing entire genomes. Thus, the shotgun method produces potentially all the genetic information of community members. The challenge to using the shotgun approach is that its analysis is challenging due to its big data size and short read lengths. To answer my research questions, our group have developed efficient methods to assemble metagenome and also extract genes of interest in the big soil shotgun sequencing data (ADD REF, SSUsearch, dignorm+partition, Xander).

Other than using better sequencing technology, we sampled the rhizosphere very close (on the roots or associated soil fragments, {\textless} 2 mm) to the roots rather than bulk soil as used in other studies 7. Similar rhizosphere sampling methods have been use in some studies 23,24. Even though bulk soil community is also influenced by the plant roots, they are more influenced by soil physical and chemical characteristics and long history of agriculture management. Rhizosphere, as the interface of soil and plant roots, is more impacted by the plant roots so better represents the selection effects of the plant. This rhizosphere sampling method better serves our goals to find the groups and genes selected by plants.

\section{Methods}

{\bf Soil samples, DNA extraction, and sequencing.}
Rhizosphere samples of corn, switchgrass and Miscanthus were collected at Great Lake Bioenergy Research Center (GLBRC) intensive Cropping System Comparison site in Kellog Biological Station (KBS) (N:xx, W:yy) in Michigan on 10/12/2012, http://data.sustainability.glbrc.org/pages/1.html. We sampled rhizospheres of each crop at seven plot areas. Each replicate was a composite of two or three plant close to the sampling spot. Roots were shaken vigorously to remove loosely attached soil, then cut off from stem and placed in plastic bags 4 degrees C. Soil closely attached to roots ({\textless} 1 mm) were collected as rhizosphere soil in lab (details see add ref Ederson). Soil left in plastic were send to measure soil chemistry. DNA was extracted with PowerSoil DNA Isolation Kit from MO BIO, USA (details see add ref Edersion) and then sent to Joint Genome Institute (JGI) for Illumina HiSeq 2500 sequencing (250 bp insert libraries and 2 x 150 bp reads).

{\bf Data preprocessing.}
Raw Illumina reads were quality trimmed using fastq-mcf (version x.x.x, http://code.google.com/p/ea-utils) with flags ``-l 50 -q 30 -w 4 -x 10 --max-ns 0 -X''. Then paired ends were merged by FLASH (add ref) using flags ``-m 10 -M 120 -x 0.20 -r 140 -f 250 -s 25''.

{\bf Diversity analysis using SSU rRNA gene.}
SSU rRNA gene fragments in shotgun metagenome are identified and an OTU table was made using fragments aligned to a 150 bp region of V4 (E.coli position: xx - yy) by SSUsearch pipeline (add ref). Ordination, richness and diversity analysis were calculated by phyloseq package in R (add ref). Venn diagram were made by in house python scripts.

{\bf Global metagenome assembly, annotation and functional diversity.}
Preprocessed reads of each crop were pooled and then assembled with digital normalization (-C 10) and partitioning pipeline (add ref) following tutorial provided in xxxx with khmer (version xx). The partitioned data were then assembled with velvet (version xx) (add ref) and SGA (version xx) (add ref). For velvet, kmer size of 29, 39, 49, 59 and 69 were used and final assembly was merged with SGA. For SGA, overlap size of 29, 39, 49, 59, and 69 were used and final assembly were clustered using cd-hit-est (-c 0.99) in CDHIT (add ref). Contigs shorter than 300 bp were discarded. Since velvet and SGA produced very similar assembly in terms of total size and longest contigs. The assembly from SGA was chosen for further analysis.

The assembly was then uploaded to MG-RAST (add ref) for gene calling and annotation. Gene calling files (with suffix .genecalling.coding.faa), clustering files (with suffix .cluster.aa90.mapping), and BLAT (add ref) tabular output files (with suffix .superblat.sims) were downloaded. BLAT hits with alignment longer than 30 amino acids, identity higher than 70\%, and E-value lower than 0.00001 were used for downstream analysis. Number of hits to each reference in MG-RAST M5NR database for each sample was summarized based on BLAT hits, gene calling IDs, and clustering info. Then M5NR IDs were converted into subsystem IDs and further subsystem ontologies using MG-RAST API (http://api.metagenomics.anl.gov/api.html). At the end, an count table of subsystem ontologies across each samples was made. Level 3 and Level 1 subsystem were used for ordination analysis with R phyloseq package (add ref). For enrichment analysis, group\_significance.py in QIIME was used to find level 3 subsystems (pathways) that significantly different between two crops ($p < 0.05, n = 7$) and those pathways with more than 10 total counts in all samples and a fold change less than 0.5 or larger than 2 were treated as sigficant.

{\bf N cycle gene assembly and diversity.}
Preprocessed reads of each crop were used as input to Xander (gene targeted metagenomic assembler) (add ref) with MAX\_JVM\_HEAP=300G, FILTER\_SIZE=40, K\_SIZE=45, genes=''nosZ nosZ\_a1 nosZ\_a2 nirS amoA\_AOA amoA\_AOB norB\_cNor norB\_qNor narG''. Output files with suffix ``.taxonabund.txt'' were used for taxonomy analysis. The abundance of each N cycle genes wre normalized by number of rplB gene (single copy gene).

{\bf Reproducibility and data accession.}
The scripts used to analyze the data are available at GitHub. Raw reads can be accessed at JGI portal with accession number: xx, yy, and zz. Assembled contigs can be accessed at MG-RAST at xx, yy, and zz.

\section{Results}

1) SSU
- beta-diversity: venn, pcoa
  - PCoA
  - shared OTU w/ and w/o weighing abundance
  - best match genus

The average read number after quality trimming and pair end assembly was about 228 millions (range from 140 millions to 285 millions) for each replicate. The reads sizes range from 50 bp to 290 bp. An average of 0.039\% of data were identified as SSU rRNA gene fragment (Table X). The number of SSU rRNA gene fragments aligned to 150 bp of V4 region is 9.4\% on average (range from 8.3\% to 10.0\%).

All samples were subsampled to the same number (4598) of SSU rRNA gene fragments and a total of 12841 OTUs are clusted. The OTU number in each replicate is 2164 on average (range from 1707 to 2355, Table X).

Ordination analysis using OTUs clustered from fragments aligned to V4 region of SSU rRNA gene showed that perennial grass (switchgrass and Miscanthus) had significantly different community structure from annual grass (corn) (AMOVA test: $p < 0.001$). The variation between corn and perennials is on PC1 and PC1 took up 48\% of total variation. Difference between Miscanthus and switchgrass were not significant and mainly varied on PC2 that only took up 8\% of total variation (Fig. X). Corn also had greater dispersion than the perennials. Venn diagrams were used to quantify similarity among communities. Three crops shared 21.5\% of their total unique OTUs, but 73.1\% of all members when abundance of each OTU was considered. At genus level, they shared 55.2\% of unique genera, but 98.1\% of all members when abundance of each genus was considered.

- richness, diversity, evenness

Microbial community in perennials rhizosphere have higher species richness (ACE) and diversity (inverse simpson) than in corn (Fig.X). Members in perennials are also more evenly distributed than in corn (rank abundance curve, Fig.X).

Corn had more Proteobacteria in rhizosphere while perennials had more Acidobacteria. Proteobacteria are mostly faster growers and Acidobacteria are mostly slow growers (add ref). Thus corn selected for faster grower while perenials selected for slow growers. Top three genera are Sphingomonas (Proteobacteria), Penicillium (Fungi), Burkholderia (Proteobacteria) in corn, Da023 (Acidobacteria), Akiw534 (Actinobacteria), Sphingomonas (proteobacteria) in Miscanthus and switchgrass. 
- domain distribution (fungi, Table X)
Bacteria, Archaea, and Eukaryota were XX\%, YY\%, and ZZ\% respectively on average across all samples. Corn (XX\%) had more fungi than perennials (XX\%).
  
2) Global assembly

- computation time

We applied digital normalization and partitioning method (ADD REF) to assemble pooled data (300 Gb) for each plant. The average computation time is 1510 cpu hours (63 days in total, Diginorm: 71h, Filterbelow: 25h, Loadgraph: 1166h, Partgraph: 205h, Mergegraph: 8h, Annograph: 27h, Extrgraph: 8h). Memory usage varies in difference steps. The peak memories used is 1 TB in digital normalization, which produced an average false positive rate of XX in kmer counting and false positive rates below 20\% are acceptable for digital normalization(ADD REF). 

- assembly stats

Our assembly method produced  XX, YY and ZZ Gb assembly with median contig size of XX, YY, ZZ and maximum contig size of XX, YY, ZZ bp for corn, Miscanthus, and switchgrass respectively (Minumum contig size cutoff: 300 bp)(Table X). When mapping quality trimmed reads to assembly, we found XX, YY, and ZZ \% reads could be mapped, suggesting XX, YY, and ZZ \% of trimmed read data were assembled in corn, Miscanthus and switchgrass respectively (Table X).

- contig comparison

We found corn was the most different among three biofuel crops.  Corn and Miscanthus had XX, corn and switchgrass had YY, and switchgrass and Miscanthus had ZZ similarity in assembled contigs. (More on size cutoff effect in discussion.)

3) Functional annotation from assembly

- PCoA with functional categories
Ordination analysis with both subsystem level 1 and 3 shows corn was different perennials and no significant difference between Miscanthus and switchgrass. The total variation explained by first two PCs are XX \% and YY \% in PCoA with level 1 subsystems and XX \% and YY \% in PCoA with level 3 subsystems (Figure X and Y).

- DE analysis with corn and Miscanthus

We found corn had more pathway (level 3 subsystems) genes in ``Stress Response'', ``Carbohydrate Metabolism'', and ``Cell Wall/Capsule'', while Miscanthus had more in ``Amino Acid Metabolism'', ``N Fixation'', ``Denitrification'', and ``Motility''.

  @add details in specific subsystems (Stree Response, Carbohydrate, and N cycle).

  @add three way tertiary plot

4) Xander assembly results

- N cycle gene abundance and taxonomy composition.

Since genes involved in nitrogen fixation and nitrification are not abundant in rhizosphere metagenome (add ref), those genes may not be well assembled in global assemblers. The annotation database that we use, SEED subsystem (add ref), does not include any nitrification genes. Thus our global assembly and annotation pipeline is not good option to analyze N cycle genes. We used Xander (ref), which is a gene targeted assembler that uses HMM (Hiden Marcov Model) of N cycle gene reference sequences to guide the assembly. The scripts for Xander assembly can be found at XXX.

N cycle genes abundances, including Nitrogen fixation genes (nifH), nitrification genes (AOA and AOB), and denitrification genes (nirK, nirS, nosZ, nosZa2, qNOR, and cNOR) were normalized by rplB abundance and shown in Figure X. NifH, nirK, nosZa2, and qNOR were significantly different between corn and Miscanthus ($p < 0.05, n = 7$). NirK, nosZ and qNOR  were significantly different between corn and switchgrass. NosZ was significantly different between Miscanthus and switchgrass.

@add alpha diversity with taxon table

@add alpha and beta diversity with OTU table

- ratios of interest

There are several pairs of genes coding for enzyme of the same process and we found a consistent pattern across all samples: AOA {\textgreater} AOB, nirK {\textgreater} nirS, nosZa2 {\textgreater} nosZ, qNOR {\textgreater} cNOR. Moreover, nitrification gene abundances (AOA and AOB) were higher than nitrogen fixation genes (nifH), and denitrification gene abundances (nirK and nirS, qNOR and cNOR, or nosZ and nosZa2) were higher than nitrification genes (AOA and AOB).

- Species abundance from Xander vs. from SSU.

- Add correlation between N2O and genes

- Add crop production data

\section{Discussion}

Biofuel is a very important sustainable energy. Large scale plantation of biofuel crops may change soil microbial community structure and thus have impact on ecosystem services, such C and N cycles. We use shotgun metagenomics to investigate the microbial community structure and function of three biofuel crops (corn, Miscanthus, and switchgrass), with a focus on N cycle genes. Shotgun metagenomics have advantage over amplicon based studies. It eliminates the primer bias and chimera and also provides all the genomic information that different tools can be applied to mine functional information of interest.

1) consistent separation of corn from perennials.

We find corn is the most different among three crops, in community structure (Fig. X and Y), assembled genomic sequences, and functional profile (Fig. Z). It is known that change of plants can change soil microbial community (add ref). Another study has shown that similar results of rhizosphere community structure of three crops using amplicons of 16S rRNA gene (add ref), but our analysis on community structure is based on SSU rRNA extracted from shotgun metagenome thus includes 18S rRNA from Eukaryota and not limited by chimera and primer bias. The ordination analyses with weighted and unweighted unifrac distance both shows separation of three crops, but the variation explained by first two axes in weighted unifrac (61.8\%) are much larger than unweighted (15.7\%), which indicates separation of crops are more significant when abundance are considered (ADD adonis test). Consistently, the big difference between percentage of shared members without (21.5\%) and with (73.1\%) being weighed by abundance at OTU or genera level (Fig. X and Y) suggests that most OTUs specific to one crop have low abundance and thus the difference among three crops are caused by abundance of shared members, not by OTUs unique to one crop. 

Our comparative metagenomic analyses of assemblies and functional profile is beyond the existing studies relying on 16S rRNA gene or a few other functional genes (ADD REF). Metagenome assembly could represent the genomic content of communities. The sequence similarity comparison of metagenome assemblies of three crops also shows corn has the most different community (Table X), but the similarities between crops are smaller than those from OTUs without considering the abundance, which might be explained by different genetic content of members within the same species (OTU). Thus the above indicates larger genetic diversity compared to species (OTU) diveristy defined by SSU rRNA gene in metagenome. Moreover, functional diveristy using subsystems (level 1 and level 3) also confirms separation of annual and perennial as shown with OTU (species diversity) and assembly (genetic diversity) and suggests that change of biofuel crop in large scale can have impact on ecosystem services, such as N and C cycling.

- data loss

Only 20\% of reads are used in the assembly. Only XX\% of genes in assembly can be assigned a function in subsystem.

2) larger dispersion in corn.

Corn also shows more variation in beta diversity among replicates (Fig. X). There are two possible explanations: 1) Corn is annual, grows from seeds and establish its root system every year, while perennials (Miscanthus and switchgrass) have more stable root systems. As a result, rhizosphere microbial communities of corn show more stochasticity (more variation among replicates). 2) Corn has been bred for corn production under intensive fertilization for a long time and some traits for recruiting beneficial microbiome may have been lost.

- Alpha diversity

Perennials have higher alpha (local) diversity (Fig. X) and also higher total OTU number (gamma diversity) when replicates are combined (Fig X). Thus change from annual to perennials as biofuel crops can increase the overall microbial diversity, which could have a significant impact on ecosystem. The (OTU) richness of perennials are not significantly higher, partly due to high variation in corn, but higher evenness (Fig. X) also contributes to the significance of higher diversity in perennials. 

According to a study on disturbance and diversity models (ADD REF, Svensson 2012), evenness should increase when disturbance are introduced. The annual life cycle of corn should encourage a more even community, which contradicts to our results (less evenness compared to perennials). Thus perennials are good at maintaining a more even community due to root exudates or agricultural management other than their their cycle such as less fertilizer usage.

@Bigger dispersion in weighted PCOA != diff species in reps (could be just abundance)

- Taxonomy

Other than difference in diversity, there are also signifcant differnce in community composition between corn and the other two perennial crops. Proteobacteria and Bacteroidetes enriched in corn are commonly considered to be copiotrophic taxa and Acidobacteria enriched in perennials are oligotrophic taxa (ADD REF, Fierer 2007, Eilers 2010, Davis 2011). Nitrogen addition can cause a shift of oligotrophic taxa (ADD REF, fierer 2012, Campbell 2010, Wessen 2010). Thus it is sensible that we have more copiotrophic taxa in corn because more fertilizer were added and more nitrogen were detected in soil loosely attached corn (Table X). Further, since copiotrophs need more carbon as well nitrogen, we predict that corn is also producing more exudates as carbon source compared to perennials, which is consistent with more ``Carbohydrates'' related pathway genes enriched in corn rhizosphere shown in functional diversity analysis (Fig. X). Further experiemnts are needed to verify this point though. Moreover, copiotrophs have relatively fast growth rate in nutrient rich environment, which explains the lower evenness in corn (ADD REF Fierer 2012).

- Higher fungal percentage in corn $\rightarrow$ C/N ratio, contradiction, more exudates from corn $\rightarrow$ copiotrophic $\rightarrow$ not selecting beneficial microbes

In addition, the higher fungal-to-bacterial ratio in corn also suggests corn provides more carbon source to the rhizosphere compared to perennials and enourages copiotrophic bacteria, which is consistent with the above. SSU rRNA gene from shotgun metagenome includes Bacteria, Archaea and Eukaryota. Therefore, we are able to get fungal-to-bacterial ratio. The ratios of three crops are all higher than the one of a bulk soil sample in KBS reported in another study (ADD REF, SSUsearch). Higher fungal-to-bacterial ratio commonly indicates higher C/N ratio since fungi biomass has higher C/N ratio than bacteria (ADD REF). The ratio here is much smaller than reported in other studies using biomass (ADD REF, ED paper). Two possible explanations are: 1) Fungi has more biomass than bacteria and some even has cells without nucleus. 2) DNA extraction method does not extract fungi DNA effectively. Moreover, Penicillium is highly enriched in corn (the third most abundant genus). It is commonly saprotroph (degrading organic matters) but hardly known as benefical to plants (to the best of our knowledge), which suggests that corn provides more carbon source to rhizosphere but not effectively recruiting beneficial members at least in this case.

- look into ``Carbohydrates'', recalcitrant (oligo) and liable

4) poor corn production - stress respons genes + Nitrate and/or nitrous oxide

Higher ``Stress responses'' pathway genes, especially ``Dessication stress'', is consistent with poor corn yield in a drough year.

5) N cycle enabled by Xander. Three mains points: 1) more nifH in perennials 2) AOA is more abundant than AOB and is the majority of Archaea 3) paralogs are ignored in other studies

- ordination of N cycle genes

OTU based ordination analyses of XX, YY, ZZ genes shows the same separation of corn and perennials as SSU rRNA gene, suggesting that nitrogen fixation, nitrification, and denitrification community composition are also changed in perennials compared to corn, following the overall community. @Discuss more depends on what genes shows separation of corn and perennials.

- quantity of N cycle genes

Perennials has more nifH than corn on average (significant for Miscanthus but not significant for switchgrass), suggesting perennials have more nitrogen fixing members. This is consistent with other studies (ADD REF, Mao 2013). Moreover, Miscanthus is the most diverse including 9 genera (7 in corn and 5 in switchgrass). Three crops are very different in community composition with Rhizobium most abundant in corn, Coraliomargarita most abundant in Miscanthus, and Novosphingobium most abundant in switchgrass (Fig. X), which suggests each crop is different in selecting the nitrogen fixing members.

More AOA than AOB in all samples suggests archaeal nitrifiers are the majority in nitrifying community in our rhizosphere soil samples, which is common in soil (ADD REF). AOA is also the majority of archaea since archaea is XX\% in total community based on SSU rRNA gene and AOA is about 1\% based on Xander results (Fig. X). Further, we expected corn to have highest percentage of nitrifying members due to addition of fertilizer, but Miscanthus have the highest AOA and AOB counts (Fig. X). @explain.

Overall there are more denitrification than nitrification or nitrogen fixation genes in all samples. Nitrite reductase (nirK and nirS) are more abundant in corn, but nitric oxide reductase (qNOR and cNOR) and nitrous oxide reductase (nosZ and nosZa2) are more abundant in perennials. @explain. 

- Paralogs are not mentioned in amplicon studies, also new knowledge.

Denitrification involves multiple steps and some steps have more than one enzymes involved.  Our results shows that different steps can indicate different conclusions (e.g. nirK vs. qNOR or nirK vs. nosZa2) and some genes are too few to represent the process rate compared paralogs coding enzymes with the same function (e.g. nirS in nitrite reduction, cNOR in nitric oxide reduction, and nosZ in nitrous oxide reduction). Most of other studies only pick one enzyme of one step to represent the process rate of whole pathway (ADD REF) and could have misleading or incomplete conclusion.

Perennials have lower nitrification/fixation, denitrification/nitrification ratios (Fig. X), suggesting perennials are better than keeping nitrogen in the soil and thus better nitrogen sustainability.

@Discuss the variation within a gene, identity distribution to the best hit. Discuss the diversity of genes with only a few taxa, e.g. AOA and AOB.

- Relation between substrate (nitrate and nitrous oxide) and N cycle gene number. 

- relative abundance of members in N cycle in whole community is new knowledge.

- rplB diversity consistent with SSU. 

The main purpose of rplB is to standardize the copy number of N cycle gene across all samples. SSU rRNA gene or total basepair in each sample are also reasonable choices, but rplB is a single copy gene and thus can stardardize the counts of N cycle genes to counts per cell, which is same as relative abundance within the whole community, a number of great interest for us to understand the member involved in N cycle and can not be easily measured. Futhre, rplB can also be used as a phylogentic marker for community structure as SSU rRNA gene with the advantage of being a single copy gene (No multiple OTUs from mutiple copies of SSU rRNA gene in the same genome). The same grouping of samples (separation of corn and perennials) and similar total number of OTUs (@need to verify) of rplB and SSU rRNA gene validates both methods and also confirms that corn and perennials have different overall community structure.

- Discuss sample time and seasonal change ignored here. 

\section{Conclusion}

- Similar to plant and insect, microbial species and functional diversity all increase under perennials.

- Corn provides more carbon source (fungal/bacterial ratio, ``Carbohydrates'' and copiotrophs) but does not effectively recruit beneficial ones (more ``Stress response'', Penicillium, poor yield). Wasting energy.

- N cycle genes

There are more nifH in Miscanthus and switchgrass than corn, and ratio of amoA/nifH is lower in perennials, suggesting better nitrogen sustainability with perennials.

\end{document}